\frontmatter												%Seitennumerierung
\pagenumbering{Roman}										%Römische Zahlen
\addtocounter{page}{2}

\newcommand{\doublesignature}[2]{%
  \parbox{\textwidth}{
    \hfill
    \parbox{7cm}{
      \centering
      \rule{6cm}{1pt}\\
      #1
    }
    \parbox{7cm}{
      \centering
      \rule{6cm}{1pt}\\
      #2
    }
  }
  \mbox{}\\
  \mbox{}\\
  \mbox{}\\
  \mbox{}\\
}

\vspace*{20pt}

\section*{Eidesstattliche Erklärung}
\label{sec:eidesstattliche-erklaerung}
Ich erkläre an Eides statt, dass ich die vorliegende Arbeit selbstständig verfasst, andere als die angegebenen
Quellen/Hilfsmittel nicht benutzt und die den benutzten Quellen wörtlich und inhaltlich entnommenen
Stellen als solche kenntlich gemacht habe.\\
\\
Arnfels, am 5. April 2018\\

\vskip 1cm

\doublesignature{Florian Greistorfer}{Marian Korošec}
\doublesignature{Thomas Test}{Peter Platzhalter}

\vskip 5cm

\clearpage

\newpage
\thispagestyle{empty}
\mbox{}

\clearpage

\section*{Danksagung}
\label{sec:danksagung}
An dieser Stelle möchten wir uns bei allen bedanken, die uns im Rahmen der Diplomarbeit unterstützt und betreut haben.\\
\\
TODO

\clearpage

\newpage
\thispagestyle{empty}
\mbox{}

\clearpage

\section*{Abstract}
\label{sec:abstract}
TODO

\section*{Zusammenfassung}
TODO

\clearpage

\newpage
\thispagestyle{empty}
\mbox{}

\clearpage

\subsection*{Gender Erklärung}
\label{sec:gender-erklaerung}
Aus Gründen der besseren Lesbarkeit wird in dieser Arbeit die Sprachform des generischen Maskulinums angewendet. Es wird an dieser Stelle darauf hingewiesen, dass die ausschließliche Verwendung der männlichen Form geschlechtsunabhängig verstanden werden soll.

\subsection*{Über dieses Dokument}
\label{sec:ueber-dokument}
Diese Arbeit wurde in \LaTeX{} verfasst. Diese Art der Dokumentation bietet gegenüber den normalen Textverarbeitungen gewisse Vorteile hinsichtlich der Formatierung und des Einbindens von Grafiken. Auch Formeln können sehr einfach und effizient angegegeben werden. Die Rohfassung des Dokuments befindet sich auf dem Arnfelser Gitweb Server der HTBLA Kaindorf Abteilung Mechatronik.

\clearpage

\newpage
\thispagestyle{empty}
\mbox{}

\clearpage

\section*{Projektteam}
\label{sec:projektteam}

\subsection*{Florian Greistorfer}
\begin{wrapfigure}[10]{l}{0.5\textwidth}
\begin{center}
  \includegraphics[width=0.35\textwidth]{fig/logoMecha}
\end{center}
\end{wrapfigure}
\mbox{}\\
\mbox{}\\
\textbf{Aufgabenbereich}:\\
\LaTeX{}\\
\textbf{Betreuer}:\\
Marian Korošec
\mbox{}\\
\mbox{}\\
\mbox{}\\
\mbox{}\\
\mbox{}\\
\mbox{}\\

\subsection*{Marian Korošec}
\begin{wrapfigure}[10]{l}{0.5\textwidth}
\begin{center}
  \includegraphics[width=0.35\textwidth]{fig/logoMecha}
\end{center}
\end{wrapfigure}
\mbox{}\\
\mbox{}\\
\textbf{Aufgabenbereich}:\\
\LaTeX{}\\
\textbf{Betreuer}:\\
Florian Greistorfer
\mbox{}\\
\mbox{}\\
\mbox{}\\
\mbox{}\\
\mbox{}\\
\newpage

\subsection*{Thomas Test}
\begin{wrapfigure}[10]{l}{0.5\textwidth}
\begin{center}
  \includegraphics[width=0.35\textwidth]{fig/logoMecha}
\end{center}
\end{wrapfigure}
\mbox{}\\
\mbox{}\\
\textbf{Aufgabenbereich}:\\
Testen\\
\textbf{Betreuer}:\\
Professor Oak
\mbox{}\\
\mbox{}\\
\mbox{}\\
\mbox{}\\
\mbox{}\\
\mbox{}\\

\subsection*{Peter Platzhalter}
\begin{wrapfigure}[10]{l}{0.5\textwidth}
\begin{center}
  \includegraphics[width=0.35\textwidth]{fig/logoMecha}
\end{center}
\end{wrapfigure}
\mbox{}\\
\mbox{}\\
\textbf{Aufgabenbereich}:\\
Platz halten\\
\textbf{Betreuer}:\\
Professor Oak
\mbox{}\\
\mbox{}\\
\mbox{}\\
\mbox{}\\
\mbox{}\\
\newpage
