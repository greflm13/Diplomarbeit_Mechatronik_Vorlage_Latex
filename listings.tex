%Style for the \inlinecode{}{} command
%command use: \inlinecode{language}{code} e.g.: \inlinecode{Java}{public static void main()}
\lstset{
	backgroundcolor=\color{codeBackGray},
    showstringspaces=false,
	keepspaces=true,
    basicstyle=\fontencoding{T1}\fontfamily{DejaVuSansMono-TLF}\fontseries{m}\selectfont\footnotesize,
    keywordstyle=\color{blue},
    commentstyle=\color[grey]{0.6},
    stringstyle=\color[RGB]{255,150,75}
}

%Java
%Model from netbeans
\definecolor{java_net_comment}{rgb}{0.586,0.586,0.586}
\definecolor{java_net_keyword}{rgb}{0,0,0.898}
\definecolor{java_net_string}{rgb}{0.805,0.480,0}
\definecolor{java_net_preprocessor}{rgb}{0,0.597,0}
\definecolor{codeBackGray}{gray}{0.98}

\lstdefinestyle{java}{ 																					% define java style
  language=Java,                 																		% the language of the code
  backgroundcolor=\color{codeBackGray},   																% choose the background color
  basicstyle=\fontencoding{T1}\fontfamily{courier}\fontseries{m}\selectfont\footnotesize,    			% the size of the fonts that are used for the code
  breakatwhitespace=false,         																		% sets if automatic breaks should only happen at whitespace
  breaklines=true,                 																		% sets automatic line breaking
  captionpos=b,                    																		% sets the caption-position to bottom
  commentstyle=\color{java_net_comment},    															% comment style
  escapeinside={(}{)},          																		% if you want to add LaTeX within your code
  extendedchars=true,             		 																% lets you use non-ASCII characters; for 8-bits encodings only, does not work with UTF-8
  frame=none,%single,	                   																% adds a frame around the code
  framexleftmargin=8mm,																					% include numbers into the frame
  keepspaces=true,                																		% keeps spaces in text, useful for keeping indentation of code (possibly needs columns=flexible)
  keywordstyle=\color{java_net_keyword},       															% keyword style
  deletekeywords=          																				% if you want to delete keywords from the given language
 {}, 
  otherkeywords={},           																			% if you want to add more keywords to the set
  numbers=left,                    																		% where to put the line-numbers; possible values are (none, left, right)
  numbersep=8pt,                   																		% how far the line-numbers are from the code
  numberstyle=\fontencoding{T1}\fontfamily{courier}\fontseries{m}\selectfont\footnotesize,				% the style that is used for the line-numbers
  rulecolor=\color{black},         																		% if not set, the frame-color may be changed on line-breaks within not-black text (e.g. comments (green here))
  showspaces=false,                																		% show spaces everywhere adding particular underscores; it overrides 'showstringspaces'
  showstringspaces=false,          																		% underline spaces within strings only
  showtabs=false,                  																		% show tabs within strings adding particular underscores
  stepnumber=1,                    																		% the step between two line-numbers. If it's 1, each line will be numbered
  stringstyle=\color{java_net_string},     																% string literal style
  tabsize=2,	                   																		% sets default tabsize to 2 spaces
  title=\lstname,                  		 																% show the filename of files included with
  literate=%
  {Ö}{{\"O}}1
  {Ä}{{\"A}}1
  {Ü}{{\"U}}1
  {ß}{{\ss}}1
  {ü}{{\"u}}1
  {ä}{{\"a}}1
  {ö}{{\"o}}1																							% escape ÖÄÜßüäö
}

%C
%Model from netbeans
\definecolor{C_net_comment}{rgb}{0.586,0.586,0.586}
\definecolor{C_net_keyword}{rgb}{0,0,0.898}
\definecolor{C_net_string}{rgb}{0.805,0.480,0}
\definecolor{C_net_preprocessor}{rgb}{0,0.597,0}
\definecolor{codeBackGray}{gray}{0.98}

\lstdefinestyle{C}{ %define c style
  language=C,                 			% the language of the code
  backgroundcolor=\color{codeBackGray},   	% choose the background color; you must add \usepackage{color} or \usepackage{xcolor}
  basicstyle=\fontencoding{T1}\fontfamily{DejaVuSansMono-TLF}\fontseries{m}\selectfont\footnotesize,        		% the size of the fonts that are used for the code
  breakatwhitespace=false,         		% sets if automatic breaks should only happen at whitespace
  breaklines=true,                 			% sets automatic line breaking
  captionpos=b,                    			% sets the caption-position to bottom
  commentstyle=\color{C_net_comment},    	% comment style
  escapeinside={(*}{*)},          			% if you want to add LaTeX within your code
  extendedchars=true,             		 	% lets you use non-ASCII characters; for 8-bits encodings only, does not work with UTF-8
  frame=none,%single,	                   				% adds a frame around the code
  framexleftmargin=8mm,					%include numbers into the frame
  keepspaces=true,                			% keeps spaces in text, useful for keeping indentation of code (possibly needs columns=flexible)
  keywordstyle=\color{C_net_keyword},       		% keyword style
  deletekeywords=          					% if you want to delete keywords from the given language
 {...}, 
  otherkeywords={*,...},           		% if you want to add more keywords to the set
  numbers=left,                    			% where to put the line-numbers; possible values are (none, left, right)
  numbersep=8pt,                   			% how far the line-numbers are from the code
  numberstyle=\fontencoding{T1}\fontfamily{DejaVuSansMono-TLF}\fontseries{m}\selectfont\footnotesize,		 % the style that is used for the line-numbers
  rulecolor=\color{black},         		% if not set, the frame-color may be changed on line-breaks within not-black text (e.g. comments (green here))
  showspaces=false,                			% show spaces everywhere adding particular underscores; it overrides 'showstringspaces'
  showstringspaces=false,          		% underline spaces within strings only
  showtabs=false,                  			% show tabs within strings adding particular underscores
  stepnumber=1,                    			% the step between two line-numbers. If it's 1, each line will be numbered
  stringstyle=\color{C_net_string},     		% string literal style
  tabsize=2,	                   					% sets default tabsize to 2 spaces
  title=\lstname,                  		 		% show the filename of files included with
  literate=
  {\#include}{{{\color{C_net_preprocessor}\#include}}}{8}
  {\#define}{{{\color{C_net_preprocessor}\#define}}}{7}
}

\newcommand{\inlinecode}[2]{\colorbox{editorGray}{\lstinline[language=#1]$#2$}}
