\documentclass
[
a4paper,															%Papierformat
11pt,																%Schriftgröße
twoside=true,														%Zweiseitig
openright,															%Neues Kapitel immer auf der rechten Seite
titlepage,															%Titelseite
headinclude,														%Seitengröße auch bei Kopfzeile
numbers=noenddot,													%Bei Kapiteln keine abschließenden Punkte
listof=numbered,													%Listingsverzeichnis
bibliography=totocnumbered,											%Literaturverzeichnis
]
{scrbook}															%Dokumenttyp
\usepackage[ngerman]{babel}											%Deutsch
\usepackage[bottom=1in,inner=1in,outer=20mm,top=20mm]{geometry}		%Ganze Seite
\usepackage{emptypage}												%Leere Seiten ohne Kopf und Fußzeile
\usepackage[headsepline]{scrlayer-scrpage}							%Kopf und Fußzeile
\pagestyle{scrheadings}												%Nummerierung in der Kopfzeile
\clearscrheadfoot													%Kopf und Fußzeile löschen
\rehead{\headmark}													%Kapitelname auf der geraden Seite innen
\ohead[\pagemark]{\pagemark}										%Seitennummerierung
\lohead{}															%Name auf der ungeraden Seite innen
\renewcommand*{\chapterpagestyle}{scrheadings}						%Kopf und Fußzeile auf Seiten mit Überschriften anders
\usepackage{graphicx}												%Bilder
\usepackage{caption}												%Tabellen Listings und Figuren mit beschriftung im Verzeichnissen
\usepackage[T1]{fontenc}											%Outputencoding
\usepackage{float}													%Plazierung von Floats (Bilder Tabellen)
\usepackage[utf8]{inputenc}											%UTF8
\usepackage{wrapfig}												%Textumfluss von Bildern, die nicht die ganze Seite Brauchen
\usepackage{setspace}												%Zeilenabstand
\usepackage{listings}												%Listings (=Code)
\usepackage{times}													%Schriftart Times Roman
\usepackage{courier}												%Schriftart Courier
\usepackage{multirow}												%Bessere Tabellenformatierung
\usepackage{array}													%Bessere Tabellenformatierung
\usepackage{xcolor}													%Farben für Codehighlighting oder ähnliches
\usepackage{appendix}												%Anhang Titelseite
\usepackage{tabularx}												%Bessere Tabellen
\usepackage{jurabib}												%Zitierung
\usepackage[bookmarks]{hyperref}									%Automatische Lesezeichen

\setlength{\parindent}{0em}											%Einrücken

\subject{\includegraphics[scale=0.7]{logoMecha.png}}
\title{Diplomarbeitvorlage}
\subtitle{HTBLA Kaindorf an der Sulm\\Grazer Straße 202, A-8430 Kaindorf an der Sulm\\Ausbildungsschwerpunkt Mechatronik und Automatisierungstechnik}
\author{Florian Greistorfer, Marian Korošec}
\date{Abgabedatum: 7.3.2018}
\publishers{Betreut von:\\Dipl.-Ing. Manfred Steiner}
\begin{document}													%Dokumentbeginn
\onehalfspace														%Zeilenabstand
\maketitle															%Titelseite
\setcounter{tocdepth}{5}											%Tiefe der Überschriften
\setcounter{secnumdepth}{5}											%Nummerierungstiefe der Überschriften
%Java
%Model from netbeans
\definecolor{java_net_comment}{rgb}{0.586,0.586,0.586}
\definecolor{java_net_keyword}{rgb}{0,0,0.898}
\definecolor{java_net_string}{rgb}{0.805,0.480,0}
\definecolor{java_net_preprocessor}{rgb}{0,0.597,0}
\definecolor{codeBackGray}{gray}{0.98}

\lstdefinestyle{java}{ 																					%define java style
  language=Java,                 																		% the language of the code
  backgroundcolor=\color{codeBackGray},   																% choose the background color
  basicstyle=\fontencoding{T1}\fontfamily{courier}\fontseries{m}\selectfont\footnotesize,    			% the size of the fonts that are used for the code
  breakatwhitespace=false,         																		% sets if automatic breaks should only happen at whitespace
  breaklines=true,                 																		% sets automatic line breaking
  captionpos=b,                    																		% sets the caption-position to bottom
  commentstyle=\color{java_net_comment},    															% comment style
  escapeinside={(}{)},          																		% if you want to add LaTeX within your code
  extendedchars=true,             		 																% lets you use non-ASCII characters; for 8-bits encodings only, does not work with UTF-8
  frame=none,%single,	                   																% adds a frame around the code
  framexleftmargin=8mm,																					%include numbers into the frame
  keepspaces=true,                																		% keeps spaces in text, useful for keeping indentation of code (possibly needs columns=flexible)
  keywordstyle=\color{java_net_keyword},       															% keyword style
  deletekeywords=          																				% if you want to delete keywords from the given language
 {}, 
  otherkeywords={},           																			% if you want to add more keywords to the set
  numbers=left,                    																		% where to put the line-numbers; possible values are (none, left, right)
  numbersep=8pt,                   																		% how far the line-numbers are from the code
  numberstyle=\fontencoding{T1}\fontfamily{courier}\fontseries{m}\selectfont\footnotesize,				% the style that is used for the line-numbers
  rulecolor=\color{black},         																		% if not set, the frame-color may be changed on line-breaks within not-black text (e.g. comments (green here))
  showspaces=false,                																		% show spaces everywhere adding particular underscores; it overrides 'showstringspaces'
  showstringspaces=false,          																		% underline spaces within strings only
  showtabs=false,                  																		% show tabs within strings adding particular underscores
  stepnumber=1,                    																		% the step between two line-numbers. If it's 1, each line will be numbered
  stringstyle=\color{java_net_string},     																% string literal style
  tabsize=2,	                   																		% sets default tabsize to 2 spaces
  title=\lstname,                  		 																% show the filename of files included with
}

\newcommand{\inlinecode}[2]{\colorbox{editorGray}{\lstinline[language=#1]$#2$}}
\frontmatter												%Seitennumerierung
\pagenumbering{Roman}										%Römische Zahlen
\addtocounter{page}{2}

\rehead{}
\ohead[\pagemark]{\pagemark}

\newcommand{\doublesignature}[1]{%
  \parbox{\textwidth}{
    \hfill
    \parbox{7cm}{
      \centering
      \rule{6cm}{1pt}\\
      Florian Greistorfer
    }
    \parbox{7cm}{
      \centering
      \rule{6cm}{1pt}\\
      Marian Korošec
    }
  }
}
\newcommand{\doublesign}[1]{%
\mbox{}\\
\mbox{}\\
\mbox{}\\
\mbox{}\\
  \parbox{\textwidth}{
    \hfill
    \parbox{7cm}{
      \centering
      \rule{6cm}{1pt}\\
      1 nicer Dude
    }
    \parbox{7cm}{
      \centering
      \rule{6cm}{1pt}\\
      1 anderer nicer Dude
    }
  }
}

\vspace*{20pt}

\section*{Eidestattliche Erklärung}
\label{sec:eidestattliche-erklaerung}
Ich erkläre an Eides statt, dass ich die vorliegende Arbeit selbstständig verfasst, andere als die angegebenen
Quellen/Hilfsmittel nicht benutzt und die den benutzten Quellen wörtlich und inhaltlich entnommenen
Stellen als solche kenntlich gemacht habe.\\
\\
Arnfels, am 5. April 2018\\

\vskip 1cm

\doublesignature{}
\doublesign{}

\vskip 5cm

\clearpage

\newpage
\thispagestyle{empty}
\mbox{}

\clearpage

\section*{Danksagung}
\label{sec:danksagung}

\clearpage

\newpage
\thispagestyle{empty}
\mbox{}

\clearpage

\section*{Abstract}
\label{sec:abstract}

\section*{Zusammenfassung}

\clearpage

\newpage
\thispagestyle{empty}
\mbox{}

\clearpage

\subsection*{Gender Erklärung}
\label{sec:gender-erklaerung}
Aus Gründen der besseren Lesbarkeit wird in dieser Arbeit die Sprachform des generischen Maskulinums angewendet. Es wird an dieser Stelle darauf hingewiesen, dass die ausschließliche Verwendung der männlichen Form geschlechtsunabhängig verstanden werden soll.

\subsection*{Über dieses Dokument}
\label{sec:ueber-dokument}
Diese Arbeit wurde in \LaTeX{} verfasst. Diese Art der Dokumentation bietet gegenüber den normalen Textverarbeitungen gewisse Vorteile hinsichtlich der Formatierung und des Einbindens von Grafiken. Auch Formeln können sehr einfach und effizient angegegeben werden.

\clearpage

\newpage
\thispagestyle{empty}
\mbox{}

\clearpage

\section*{Projektteam}
\label{sec:projektteam}

\subsection*{Florian Greistorfer}
\begin{wrapfigure}[12]{l}{0.5\textwidth}
\begin{center}
  \includegraphics[width=0.35\textwidth]{logoMecha}
\end{center}
\end{wrapfigure}
\mbox{}\\
\mbox{}\\
\mbox{}\\
\mbox{}\\
\mbox{}\\
\textbf{Aufgabenbereich}:\\
Latex\\
\textbf{Betreuer}:\\
Marian Korošec
\mbox{}\\
\mbox{}\\
\mbox{}\\
\mbox{}\\
\mbox{}\\
\mbox{}\\

\subsection*{Marian Korošec}
\begin{wrapfigure}[15]{l}{0.5\textwidth}
\begin{center}
  \includegraphics[width=0.35\textwidth]{LogoMecha}
\end{center}
\end{wrapfigure}
\mbox{}\\
\mbox{}\\
\mbox{}\\
\mbox{}\\
\mbox{}\\
\mbox{}\\
\textbf{Aufgabenbereich}:\\
Latex\\
\textbf{Betreuer}:\\
Florian Greistorfer
\mbox{}\\
\mbox{}\\
\mbox{}\\
\mbox{}\\
\mbox{}\\
\newpage

\subsection*{1 nicer Dude}
\begin{wrapfigure}[12]{l}{0.5\textwidth}
\begin{center}
  \includegraphics[width=0.35\textwidth]{logoMecha}
\end{center}
\end{wrapfigure}
\mbox{}\\
\mbox{}\\
\mbox{}\\
\mbox{}\\
\mbox{}\\
\mbox{}\\
\textbf{Aufgabenbereich}:\\
Exisitieren\\
\textbf{Betreuer}:\\
Ich
\mbox{}\\
\mbox{}\\
\mbox{}\\
\mbox{}\\
\mbox{}\\

\subsection*{1 anderer nicer Dude}
\begin{wrapfigure}[15]{l}{0.5\textwidth}
\begin{center}
  \includegraphics[width=0.35\textwidth]{logoMecha}
\end{center}
\end{wrapfigure}
\mbox{}\\
\mbox{}\\
\mbox{}\\
\mbox{}\\
\mbox{}\\
\textbf{Aufgabenbereich}:\\
nichjt existieren\\
Mechianik\\
\textbf{Betreuer}:\\
Du														%Datei importieren
\tableofcontents													%Inhaltsverzeichnis
\mainmatter															%Seitennummerierung
\chapter{Beispiele}
\section{Beispiel mit Code und Bildern}
\subsection{Code und Bilder}
\subsubsection{Sogar mit Textumfluss}
\paragraph{Bilder}

\begin{wrapfigure}{r}{0.7\textwidth}
\vspace{-30pt}
  \begin{center}
    \includegraphics[width=0.7\textwidth]{logoMecha}
  \end{center}
  \caption{Bild mit Textumfluss}
  \label{Bild mit Textumfluss}
  \vspace{-10pt}
\end{wrapfigure}

Hier muss Text stehen, sonst wird von unten alles verschoben. Sollte nicht genug Text vorhanden sein muss mit dem vspace Befehl das Format angepasst werden. Mit dem vspace Kommando muss solange herumgspielt werden, bis das Format passt. Sollte nicht genug Platz gebraucht werden wird der wrapfigure Befehl auch weiter unten alles auf die halbe Seite verringen. Verweisen auf die Bilder \ref{Bild mit Textumfluss} und \ref{Bild über ganze Seitenbreite} kann man mit dem Befehl \textbackslash{}ref\{\}

\begin{figure}[H]
      \includegraphics[width=1\textwidth]{logoMecha}
      \caption{Bild über ganze Seitenbreite}
      \label{Bild über ganze Seitenbreite}
\end{figure}
\newpage

\paragraph{Code} Hier muss Text stehen, sonst wird der Paragraph nicht angezeigt.

\begin{lstlisting}[style=java,caption=Codebeispiel]
package u04t;

import ue04.Bauteil;
import ue04.Kondensator;
import ue04.Placeable;
import ue04.PlaceableWiderstand;
import ue04.Spule;
import ue04.Widerstand;


/**
 *
 * @author sx
 */
public class Ue04
{
  public static void main (String[] args)
  {
    Widerstand r1 = new Widerstand(1000, 10);
    Kondensator c1 = new Kondensator(0.001, 1, r1.getCurrent());
    Spule l1 = new Spule (0.01, 5.4, c1.getCurrent());
    
    Bauteil [] bauteile = new Bauteil [4];
    bauteile[0] = r1;
    bauteile[1] = c1;
    bauteile[2] = l1;
    bauteile[3] = new PlaceableWiderstand(100, 2, 10, 12.3);
    
    double totalEnergy = 0.0;
    for (Bauteil b : bauteile)
    {
      System.out.print(b);
      if (b instanceof Placeable)
      {
        Placeable p = (Placeable)b;
        System.out.print(" x=" + p.getX() + "  y=" + p.getY());
      }
      System.out.println();
      
      totalEnergy += b.getEnergy();
    }
    System.out.println("Gesamtenergie: " + totalEnergy + "J");
    
  }
}

\end{lstlisting}
\newpage

\section{Tabellenbeispiel}

Tabelle \ref{Tabellenbeispiel} ist mit einem Onlinegenerator erzeugt, da Tabellen anstrengend sind.

\begin{table}[]
\resizebox{0.5\textwidth}{!}{%
\begin{tabular}{|c|c|c|c|c|}
\hline
\textit{\textbf{1}} & \textit{\textbf{2}}  & \textit{\textbf{3}}  & \textit{\textbf{4}}  & \textit{\textbf{5}}  \\ \hline
\textit{\textbf{2}} & \textit{\textbf{4}}  & \textit{\textbf{6}}  & \textit{\textbf{8}}  & \textit{\textbf{10}} \\ \hline
\textit{\textbf{3}} & \textit{\textbf{6}}  & \textit{\textbf{9}}  & \textit{\textbf{12}} & \textit{\textbf{15}} \\ \hline
\textit{\textbf{4}} & \textit{\textbf{8}}  & \textit{\textbf{12}} & \textit{\textbf{16}} & \textit{\textbf{20}} \\ \hline
\textit{\textbf{5}} & \textit{\textbf{10}} & \textit{\textbf{15}} & \textit{\textbf{20}} & \textit{\textbf{25}} \\ \hline
\end{tabular}%
}
\caption{Tabellenbeispiel}
\label{Tabellenbeispiel}
\end{table}
												%Datei importieren
\renewcommand\appendixname{Anhang}
\renewcommand\appendixpagename{Anhang}
\renewcommand\appendixtocname{Anhang}

\lohead{}

\appendix
\begingroup
\makeatletter
\let\ps@plain\ps@empty
\appendixpage
\makeatother
\endgroup

\listoffigures
\listoftables
\lstlistoflistings
\bibliography{Literaturverzeichnis}
\bibliographystyle{jurabib}													%Datei importieren
\end{document}